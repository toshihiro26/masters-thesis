\chapter{提案手法} \label{concept}
本研究では以上を踏まえた自然なタイミングでの動作生成を反映した柔軟な患者動作生成を行う.
\par
患者役であるシミュレータは医師役であるユーザの発話を受け取り,その発話に適した応答を生成し出力する.この発話と応答のペアである対話ログを記録される.

\section{}
\begin{figure}[p]
\end{figure}

\par 
ユーザへの患者動作の提示までの流れを図\ref{process_present_img}に示す.
本システムでは,2つのモジュールに分けて動作生成までを行う.

\subsection{判断モジュール}\label{decision_module}

\subsection{生成モジュール}\label{generate_module}




