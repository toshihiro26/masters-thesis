\chapter{提案手法} \label{concept}
今回の研究における医療面接シミュレータにはシナリオという患者情報が記載されたものが与えられる.
このシナリオには模擬患者役の主訴や家族歴,演技の指針などの情報が記載されており,シミュレータはこのシナリオに記載されている模擬患者に関するペルソナ情報をもとにユーザとの対話を行う.
\par
本研究ではそのシミュレータとユーザとのやり取りを記録した対話ログと患者情報を記載したシナリオをもとに自然な患者動作の生成を行う。

\section{}
\begin{figure}[p]
\end{figure}

\par 
ユーザへの患者動作の提示までの流れを図\ref{process_present_img}に示す.
本システムでは,2つのモジュールに分けて動作生成までを行う.

\subsection{判断モジュール}\label{decision_module}
本モジュールでは,対話システムに存在する対話ログを入力とし,その対話ログが策定した動作生成のルールと照らし合わせたときに当てはまるかどうかについて分類し,動作生成するべきと判断した 対話ログを動作生成モジュールに出力する。
\par
本モジュールでの流れを図\ref{process_decision}に提示する.

\section{}
\begin{figure}[p]
\end{figure}

分類にはLLMのGPT4-previewを利用する. 
本研究における動作をするタイミングとして,
\centerline{「模擬患者に関する病気に関連する内容や単語」}
対話ログが出現した際に患者動作を行う自然なタイミングであるとするので,対象の対話ログを検出するようにプロンプトを入力した.
図\ref{decision_prompt}に提示する.

\section{}
\begin{figure}[p]
\end{figure}

\subsection{生成モジュール}\label{generate_module}
 本モジュールでは,シナリオから事前に患者動作を生成し,動作判断モジュールから出力された対話ログを入力として,その対話ログに適した患者動作を選択した後,その患者動作を出力する。
 \par
本モジュールでの流れを図\ref{process_generate}に提示する.

\section{}
\begin{figure}[p]
\end{figure}

\subsubsection{患者元画像生成}\label{generate_base_img}
医療面接で与えられるシナリオに記載されているペルソナ情報を元にその対話に適したエージェント画像を生成する。
//エージェント画像生成までの流れを図\ref{generate_base_img}に示す.
画像生成にはDALLE-3を活用して生成を行う。DALLE-3はOpenaiが公開している画像生成AIであり,それまでのテキストから画像への生成を行うシステムでは単語や説明を無視する傾向があり、利用者は迅速なエンジニアリングを学ぶ必要があったが,DALLE-3は,提供されたテキストに正確に準拠した画像を生成することが可能になった。
\par
生成する際のプロンプトには,シナリオにおけるペルソナ情報を日本語で入力する.図\ref{generate_base_img_prompt}に提示する.
DALLE-3では画像生成の前に入力されたプロンプトを適当な形に補足したうえで画像を生成するため,入力プロンプトが同じ内容でも生成される画像はある程度ぶれて生成される。よって,生成されたエージェント画像をシステムのユーザがシナリオの再現画像エロして適当がどうかを判断し,適当でないなら再度画像を生成する。

\subsubsection{患者動作ベースの生成}\label{generate_correction_img}
DALLE-3で生成した画像については,その姿や姿勢にかなりの違いがある。そのため,以降に記述する患者動作をエージェント画像に適用する際に,その姿勢や画像が大きくゆがんでしまう場合がある。そのため,以降の処理の際に安定した患者動作を生成させるために,エージェント画像を一度同一の姿勢になるように再度画像を変化させる。 
\par
エージェント画像の姿勢の変換には画像生成AIであるStable_Diffusionの拡張機能であるIP-adapterとControl-NetのOpenposeを活用して行う.IP Adapterは画像をプロンプトの代替として使う技術であり,通常はテキスト入力を通して生成画像を制御するが,IP-Adapterではテキストと画像を別々に処理してから後でその2つを統合することにより,テキストと画像の両方の特徴を併せ持った画像の生成が可能となる。 Openposeは画像に写っている人間の姿勢を推定する技術であり,人間の姿勢を関節を線でつないだ棒人間として表現しそこから画像を生成します.これによって元画像のポーズをかなり正確に再現することができる.
\par
今回使用したOpenposeの画像を図\ref{Openpose_for_generate}に記載する.この二つを併用し,特定の姿勢を取らせて,元のエージェント画像の特徴を反映させたエージェント画像の生成を行う.

患者動作のベースの作成には,Stable_Diffusionを利用する。Stable_Diffusionには,拡張機能にContorlnetがあり,画像から抽出したポーズや構図といった特徴量をもとに、その特徴から大きく外れる画像を生成しないように画像に制限を加えます。そうすることで抽出した特徴に基づいて画像を生成することが可能になり,その機能の一部にOpenposeとIP-adapterが存在する.以下に生成に関する一連の流れを記載する.

\begin{enumerate}
   \item Openposeに利用する姿勢を実際に撮影し,その撮影画像をもとにエージェント画像の編集のためのOpenpose画像に変換
   \item Stable_Diffusionの入力にOpenpose画像およびエージェント元画像と”sitting on the chair”というプロンプトを記述し,ベース画像を生成
\end{enumerate}

\subsubsection{患者動作の合成}\label{generate_motion}
エージェントの画像の姿勢を固定した画像に対して,患者動作に関するモーションを付与するさせることで,患者動作を行うエージェント動画を生成する. エージェント画像への患者動作の付与にはMagicanimateを活用して行う. 
Magicanimateは参照画像とモーションシーケンスからビデオを生成するImage Animation技術であり,今回の研究ではエージェント画像と患者動作モーションビデオを入力することで患者動作動画を生成する. 
\par
ただし,利用するうえで以下のような問題点がある.元画像とかけ離れた姿勢の患者動作モーションビデオの場合,患者動作モーションビデオの姿勢に合わせようとするので姿が極端に変化する. この問題点の解決方法として,エージェント画像の極端な変化を軽減するためにエージェント画像の姿勢処理を行った.この処理を行うことによって,基本的には患者動作モーションビデオと同じ画角と姿勢のエージェント画像が生成できるため,エージェント画像への自然な動作付与をすることができる. また,生成した患者動作動画に対してそれぞれの動作に関する説明文を付与する.

\subsubsection{患者動作の選択}\label{select_motion}
入力された対話ログに対して,その対話ログがどの患者動作をつけることが最適かについて選択する. LLMに対して,対象の対話ログと患者動作の説明文を入力し,その対話ログに最適な患者動作の説明文を選択させ出力させる.
入力するプロンプトは図\ref{select_prompt}に提示する.
