\section{患者動作の自然なタイミングでの生成}\label{formulation_system}
本研究を進めるにあたり,患者動作とは医者の発話に対してシステムが応答する際に身体的な部位を抑える動作のことと定義する. 

\subsection{患者動作生成の判断}\label{}
自然なタイミングを決める前に患者動作でそもそも対話中における動作について考える.
\par
対話を行う上で動作は情報を伝えるうえで重要な要素である.その役割として以下の二つがある。
 \begin{itemize}
  \item 対話の補足・・・言葉だけでは伝えられないことを伝える
  \item 対話の強調・・・話したい内容を強調する
 \end{itemize}

ただし,すべての対話に対して動作を付与するわけではなく,すべての対話に動作を付与した場合対話自体が不自然なものになる.よって,自然なタイミングでの動作生成を行う際には,何らかのルールをもって対話に動作をつけるかどうかの判断をする必要がある.このルールを考えるうえで,動作と対話の関係性について注目した.
\par
対話と動作の関係性の仮説として,二つ挙げられている. 一つ目は”言語意味仮説”,二つ目は”インターフェース仮説”である.言語意味仮説は表現したい情報を言語化した後,その言語中に現れる単語の意味や表象に基づいて生成されるという仮説である. インターフェース仮説は表現したい空間や運動的なイメージを言語化しやすいように加工した心的表現に基づき,発話に伴う動作は生成されるという仮説である.  
今回の研究における医療面接という場面においては,模擬患者が行う動作としては基本的には言葉だけでは説明できないもしくは説明しづらい言葉をより詳しく説明するための動作であり,その説明に位置などの情報を必要としていないため,”単語意味仮説”の元の動作であることに近い.
\par
ただし,この仮説だけでは医療面接における動作のタイミングとしてはあまり自然でない.例として、「医師からの質問に対して,ある人物の腹痛を訴える」という場面を想定する.この場面の際に,自身について述べる場合と家族について述べる場合で動作をつけるか分かれると考えた.
\par
自身の場合は,自身のお腹が痛いという情報をより詳細に伝えるために「お腹を押さえる」という動作をしたほうが自然だが,家族の場合は「お腹を押さえる」という動作をつけないほうが対話においては自然である。
このような違いがあるため,単純に単語だけで動作をつける自然なタイミングであるかを判断することは難しい.
では,どういった条件で対話における自然な動作タイミングであるいえるのか.
\par
次に動作の機能について注目した.動作に与えられている機能として,”自己指向的機能”と”他者指向的機能”がある.   
\par
自己指向的機能は話し手の会話促進をするための機能であり,イメージの検討や言葉の選択を促す認知的機能である.一方で,他者指向的機能は聞き手の理解を助けるための機能であり,社会的な機能とも呼ばれている.
言い換えれば,自己指向的機能は話し手にとって言語化しにくい概念,また話しての語彙力を手助けする機能であり,自分の中の状況や考えをより具体的に相手に理解してもらう際に発生する.考えられる状況として,生徒が先生にわからない単語についての説明をする際に利用される動作の機能といえる.
\par
また,他者指向的機能は話し手が内容や言葉の意味を分かりやすく解釈させる機能であり,対話の中の内容や考えをよりわかりやすく相手に理解してもらう際に発生する。考えられる状況として,先生が生徒にわかりやすく説明する際に利用される際の動作の機能といえる.
この二つの機能を見たとき,動作をする立場としての違いはあるが共通して言えるのは内容や言葉について「理解」をしてもらうことだと言え,動作の本質は「理解」を求めることだと考えた.
よって,動作を行う自然なタイミングとは
\centerline{「話している内容や単語に対して特定の立場に関して理解を求める場合」}
であると仮定した.

\subsection{動作生成のルールの制定}\label{}
医療面接の場面での自然な動作生成について考える.
医療面接での動作の役割をより詳細に見たとき,例として挙げる「お腹を押さえる」には以下の二つの役割があるとされている.

 \begin{itemize}
  \item 対話の補足・・・医師に説明するには知識に差がある,もしくは語彙力不足である
  \item 対話の強調・・・その症状の度合いをより詳細に説明する
 \end{itemize}
\par
これらの役割における動作の機能としては自己指向的機能であり,模擬患者自身にとって自分の中の状況や考えをより具体的に相手に理解してもらうものであり,この場面における特定の立場は「模擬患者」であると言える.また,理解してもらいたい内容や単語は模擬患者が説明したいこと,自身の病気に関連している内容や単語である.

以上より,医療面接における動作を行う自然なタイミングは
  \centerline{「模擬患者に関する病気に関連する内容や単語」}
が出現するタイミングであるとする.
