\section{実験・結果}\label{result}

\subsection{実験}
\subsubsection*{動作判断モジュール}
実験に使用する対話ログには,医学生と模擬患者の医療面接の場面を書き起こした対話ログを活用.
対話ログは合計で?個入力し,対話ログに対してLLMがどのような判断を行ったのかをまとめた.

\subsubsection*{動作生成モジュール}

\subsection{結果}
\subsubsection*{対話ログに関する分析結果と人手での対話付与の比較}
- 事前実験で実際の対話動画とどのような違いがあったのかを確認する.
実対話とLLMとの比較項目  (動作の有無)
- true-positive(実対話:,LLM:あり)
- false-positive(実対話:あり,LLM:あり)
- true-negative(実対話:あり,LLM:なし)
- false-negative(実対話:なし,LLM:なし)

結果の比較方法
対象の対話ログを実際に人手でラベル付け
どの程度の割合でログを引っ張り出せるのか
数値ベースで確認

中身見てどのようなものが判定を受けてたのかを掲載.

どの程度の信頼性があるのか




一部を抜粋してうまくいったものとうまくいかなかったものを比較する.

\subsubsection*{生成された患者動作}
それぞれの実験での動作の一覧を記載する()


\subsubsection*{対話ログに対する患者動作選択}
