\section{実験・結果}\label{result}

\subsection{実験}
\subsubsection*{動作判断モジュール}
実験に使用する対話ログには,医学生と模擬患者の医療面接の場面を書き起こした対話ログを活用.
対話ログは合計で?個入力し,対話ログに対してLLMがどのような判断を行ったのかをまとめた.

\subsubsection*{動作生成モジュール}

\section{結果}
対話ログにタグ付けをするうえで,実際の医療面接の動画と書き起こしの対話ログをもとにどのようなタイミングで動作を行っているのかを調べた.

\subsection{対話ログの判別結果}
LLMの判別結果については,入力しプロンプトによって生成されうる対話ログの識別結果を網羅するため,同じプロンプトで3回結果を生成し,それらの生成結果をかぶりがなくなるようにまとめ,それぞれの対話ログにおける判別結果とした.

この判別結果をもとに全体の実対話のタグ付けとLLMとの比較をした結果は以下のようになった.
\begin{tabular}{|l|r|r|} \hline
True-positive & True - Negative & False - Positive \\hline
131 & 85 & 20 \\hline
\end {tabular}

半分の程度の識別結果は,人手によるタグ付けと同じ結果が得られた.ただし,判別にかけた対話ログによっては人手のタグ付けとはかなり異なった判別結果が生成された.

\subsubsection {識別結果の詳細}
今回の判別結果で生成された対話ログには主に2つの種類がある.
1つ目が,直接主訴に関する単語が出現する対話ログであり,医療面接における自身の症状や不安を医師に伝える説明の際に主訴に関する対話ログに動作を付与する判別結果が出た.検出された対話ログの数は以下のとおりである.

”判別結果”  (主訴:)
  対象対話ログ  :
         理由  :
    適当な動作 :

この種類の対話ログに関しては,LLMでの判別結果と人手でタグ付けしたものがおおよそ一致しており,そのほとんどが検出できた.

2つ目は,主訴に関する単語は出現していないが主訴の症状に関する対話ログであり,医師の質問に対して主訴に直接関わる応答文の対話ログに判別された.

”判別結果”  (主訴:)
  対象対話ログ  :
         理由  :
    適当な動作 :

この種類の対話ログに関してはLLMで判別された対話ログのうち,人手でタグ付けしたものと比較したときに,タグ付けされたものと一致しているものと一致しないものが存在した.

3つ目は,直接主訴に関係していないが主訴に関連していると思われる症状や関連情報の質問に対する応答文の対話ログに判別された.

”判別結果”  (主訴:)
  対象対話ログ  :
         理由  :
    適当な動作 :

基本的に判別比較における"True - Negative"に分類されたのはこの種類の対話ログであり,人手でのタグ付け作業においても,実際の医療面接の動画と比較して,動作をつける必要がないと判断したものにも,判別結果が生成された.

但し,以下の様な判別結果も生成された.

\subsubsection*{生成された患者動作}
それぞれの実験での動作の一覧を記載する()


\subsubsection*{対話ログに対する患者動作選択}
