\section{実験・結果}\label{result}

\subsection{実験}
\subsubsection*{動作判断モジュール}
実験に使用する対話ログには,医学生と模擬患者の医療面接の場面を書き起こした対話ログを活用.
対話ログは合計で?個入力し,対話ログに対してLLMがどのような判断を行ったのかをまとめた.

\subsubsection*{動作生成モジュール}
今回の実験において,エージェント元画像を生成する際にはシナリオに記載されている以下のペルソナ情報を使用した.
\centerline{ペルソナ:年齢,性別}

ペルソナ情報の選択理由として,今回の研究ではシナリオから模擬患者に関するエージェント画像を生成するが,シナリオに記載されているペルソナ情報はシナリオによって異なっており,職業や性格,容姿といったペルソナ情報を確実に抽出できるとは限らないため,すべてのシナリオで共通しているペルソナ情報としてこの二つのペルソナ情報を選択した.


また,生成した患者動作は医療面接の場面における模擬患者が行う動作として以下の7つを選択した.
 \begin{itemize}
  \item 頭
  \item 首
  \item 肩
  \item 胸
  \item 腹
  \item 腰
  \item 膝
 \end{itemize}
これらの動作は模擬患者が自身の主訴に関連した症状を医師に伝える時に主にとる動作であり,主訴の症状を伝えるという観点からこの7つに設定した.
加えて,実際に撮影した8つを動画をシステムに入力して患者動作モーションビデオを作成する.患者動作モーションビデオにはDenseposeを利用しており,8つの動作とそれに対応したDenseposeを生成した.
それぞれ患者動作モーションビデオを図\ref{Densepose_img}に提示する. 

\subsection{結果}
\subsubsection*{対話ログに関する分析結果と人手での対話付与の比較}

\subsubsection*{生成された患者動作}

\subsubsection*{対話ログに対する患者動作選択}
